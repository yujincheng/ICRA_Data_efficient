Using a team of robots to explore and build the map of an unknown environment is the basic task of many multi-robot applications, such as self-driving cars, home service robots, and multi-robot rescue.
In terms of perception, the robots in this team need to do Distributed Simultaneous Localization and Mapping (DSLAM) to share their understanding of the environment, build a global map together, and provide the location of each robot in this map.
In terms of decision, the robots need to direct themself to explore unknown space, thus expanding the known and explored area of a map which is updated while doing the DSLAM.

It is a common practice to use external positioning signals (e.g. GPS, motion capture) to provide the location of the robots. 
Yet in communication constrained environments, such as cave exploration and building inspection, the external positioning signals will be shielded. 
Some previous works implement DSLAM systems \cite{cieslewski2018data,lajoie2020door} to provide the location in these nonideal environments. 
The basic idea of these DSLAM is similar: 1) adopting single-robot SLAM system to provide intra-robot location, 2) sharing \& matching place recognition infomation to detect the inter-robot loop closures, 3) transmitting the sensor data related to the recognized same place to calculate relative poses and 4) merging \& optimizing the single-robot SLAM results according to the relative poses. 
These systems use camera image to extract a sophisticated feature vector to encode the place information. 
% Computing the distance between feature vectors can know if robots experience the same place. 
These feature vectors can be generated form the handcrafted method \cite{jegou2014triang} or emerging neural-network-based method \cite{radenovic2018fine, arandjelovic2016netvlad}. 
These image-based  method is communication-costing to share the feature vectors. 
Besides sharing place information, the sensor data, which is also cemera image and communication-costing, need to be transmitted between robots to get the relative pose.

Frontier based exploration strategies are usually used for robotic exploration for both single-robot  and multi-robot exploration \cite{senarathne2013efficient, umari2017autonomous, orvsulic2019efficient}. 